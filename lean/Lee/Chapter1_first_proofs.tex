\documentclass[11pt]{article}

\usepackage[T1]{fontenc}
\usepackage{lmodern}
\usepackage{microtype}
\usepackage{amsmath,amssymb,amsthm}

\title{Chapter 1 (Lee) --- First Proofs}
\author{}
\date{}

\newcommand{\R}{\mathbb{R}}
\newcommand{\Rn}{\R^n}
\newcommand{\Homeo}{\cong}
\newcommand{\Ball}[2]{B(#1,#2)}

\begin{document}
\maketitle

\paragraph{Source.}
This is a LaTeX rewrite of the first completed proof in \texttt{lean/Lee/Chapter1.lean}:
theorem \texttt{exercise\_1\_1}.

\section{Setup}
Let $M$ be a topological space and let $n\in\mathbb{N}$. In the Lean file, $\Rn$ is represented as
the function type $\mathrm{Fin}(n)\to \R$.

\section{Definitions}

\subsection*{Locally Euclidean}
We say $M$ is \emph{locally Euclidean of dimension $n$} if for every $p\in M$ there exist an open
neighborhood $U\subseteq M$ with $p\in U$, an open set $V\subseteq \Rn$, and a homeomorphism
$\varphi:U\Homeo V$.

\subsection*{Ball variant}
We say $M$ is \emph{locally Euclidean by balls} if for every $p\in M$ there exist an open
neighborhood $U\subseteq M$ with $p\in U$, a center $x\in \Rn$, a radius $r\in\R$, and a
homeomorphism $\varphi:U\Homeo \Ball{x}{r}$, where $\Ball{x}{r}$ denotes the open metric ball.

\subsection*{$\Rn$ variant}
We say $M$ is \emph{locally Euclidean by $\Rn$} if for every $p\in M$ there exist an open
neighborhood $U\subseteq M$ with $p\in U$ and a homeomorphism $\varphi:U\Homeo \Rn$.

\section{Exercise 1.1}
The Lean statement is the conjunction of the following equivalences:
\[
\mathrm{IsLocallyEuclidean}(M,n)\iff \mathrm{IsLocallyEuclideanBall}(M,n),
\qquad
\mathrm{IsLocallyEuclidean}(M,n)\iff \mathrm{IsLocallyEuclideanRn}(M,n).
\]

\subsection{Part A: $\mathrm{IsLocallyEuclidean}\iff \mathrm{IsLocallyEuclideanBall}$}

\subsubsection*{Direction 1: locally Euclidean $\Rightarrow$ ball version}
Fix $p\in M$ and assume $M$ is locally Euclidean. Choose $U\subseteq M$ open with $p\in U$, an open
set $V\subseteq \Rn$, and a homeomorphism $\varphi:U\Homeo V$.

Since $V$ is open and $\varphi(p)\in V$, there exists $\varepsilon>0$ such that
\[
\Ball{\varphi(p)}{\varepsilon}\subseteq V.
\]
(In the Lean proof this is obtained from \texttt{Metric.isOpen\_iff}.)

Define
\[
U':=\varphi^{-1}\bigl(\Ball{\varphi(p)}{\varepsilon}\bigr).
\]
Then $U'$ is open (preimage of an open ball under the continuous map $\varphi^{-1}$) and $p\in U'$
because $\varphi(p)\in \Ball{\varphi(p)}{\varepsilon}$.

Restricting $\varphi$ to $U'$ yields a homeomorphism
\[
\varphi':U'\Homeo \Ball{\varphi(p)}{\varepsilon},
\]
matching the ball definition. (Lean constructs this with \texttt{$\varphi$.preimageHomeomorph}.)

\subsubsection*{Direction 2: ball version $\Rightarrow$ locally Euclidean}
Fix $p\in M$ and assume there exist $U\subseteq M$ open with $p\in U$ and a homeomorphism
$\varphi:U\Homeo \Ball{x}{r}$ for some $x\in\Rn$ and $r\in\R$.
Let $V:=\Ball{x}{r}$, which is open in $\Rn$. Then $(U,V,\varphi)$ witnesses the locally Euclidean
definition.

\subsection{Part B: $\mathrm{IsLocallyEuclidean}\iff \mathrm{IsLocallyEuclideanRn}$}

\subsubsection*{Direction 1: locally Euclidean $\Rightarrow$ $\Rn$ version}
Fix $p\in M$ and assume $M$ is locally Euclidean. By Part~A, choose an open neighborhood $U$ of $p$
and a homeomorphism $\varphi:U\Homeo \Ball{x}{r}$.

From $p\in U$ we have $\varphi(p)\in \Ball{x}{r}$, i.e.\ $d(\varphi(p),x)<r$. Since distances are
nonnegative, this implies $r>0$.

There is a standard homeomorphism between an open ball and all of $\Rn$ (for $r>0$):
\[
\Ball{x}{r}\Homeo \Rn.
\]
In the Lean proof this is \texttt{Homeomorph.univBall x r hr} (used with \texttt{.symm}).

\paragraph{One explicit model.}
For the unit ball $\Ball{0}{1}\subseteq\Rn$, a standard choice is
\[
f:\Ball{0}{1}\to\Rn,\qquad f(u)=\frac{u}{1-\lVert u\rVert},
\]
with inverse
\[
f^{-1}:\Rn\to\Ball{0}{1},\qquad f^{-1}(v)=\frac{v}{1+\lVert v\rVert}.
\]
For a general ball $\Ball{x}{r}$ with $r>0$, precompose with translation and scaling
$u=(y-x)/r$.

Composing with $\varphi$ gives $U\Homeo \Rn$, proving the $\Rn$ variant.

\subsubsection*{Direction 2: $\Rn$ version $\Rightarrow$ locally Euclidean}
If $p\in U\subseteq M$ with $U$ open and $U\Homeo \Rn$, then taking $V=\Rn$ in the original
definition immediately yields that $M$ is locally Euclidean.

\end{document}
