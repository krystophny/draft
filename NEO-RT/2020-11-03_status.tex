%% LyX 2.3.2 created this file.  For more info, see http://www.lyx.org/.
%% Do not edit unless you really know what you are doing.
\documentclass[17pt,english]{foils}
\usepackage{tgheros}
\usepackage{beramono}
\usepackage[T1]{fontenc}
\usepackage[latin9]{inputenc}
\usepackage[landscape,a4paper]{geometry}
\geometry{verbose,tmargin=2cm,bmargin=3cm,lmargin=1.5cm,rmargin=1.5cm}
\setcounter{secnumdepth}{1}
\setcounter{tocdepth}{1}
\setlength{\parskip}{\smallskipamount}
\setlength{\parindent}{0pt}
\usepackage{amsmath}
\usepackage{babel}
\begin{document}
\global\long\def\d{\mathrm{d}}%


\MyLogo{~}
\title{~\\
~\\
Status NEO-RT developments}
\author{Christopher Albert, Sergei Kasilov}
\maketitle

\foilhead{The problem to solve}
\begin{enumerate}
\item Particle flux is given via integral
\begin{align}
\text{\ensuremath{\Gamma}}_{\mathrm{box}}(r)= & -2\pi^{3}\sum_{\sigma=\pm}\sum_{mn}\int\d H_{0}\int\d J_{\perp}\int\d p_{\varphi}|H_{mn}|^{2}\delta(m\omega_{b}+n\Omega_{\varphi})\partial_{mn}f_{0}\nonumber \\
 & \times\int_{0}^{\tau_{b}}\d\tau\,\Theta(r-r(R,Z))\partial_{mn}r(R,Z).
\end{align}
Here $m$ is the bounce harmonic, and $n$ is the toroidal harmonic
(symmetry angle) and
\begin{equation}
\partial_{mn}a=\mathbf{m}\cdot\nabla_{\mathbf{J}}a=m\omega_{b}\left(\frac{\partial a}{\partial H_{0}}\right)_{\boldsymbol{\theta}}+n\left(\frac{\partial a}{\partial p_{\varphi}}\right)_{\boldsymbol{\theta}}.
\end{equation}
\item Hamiltonian perturbation is given via bounce average
\begin{equation}
H_{mn}=\left\langle H_{n}(R,Z)\,e^{-i(m\omega_{b}\tau+n(\varphi-\Omega_{\varphi}\tau))}\right\rangle _{b}
\end{equation}
with
\begin{equation}
H_{n}(R,Z)=\left(J_{\perp}\omega_{c0}(R,Z)+m_{\alpha}v_{\parallel0}^{2}(R,Z)\right)\frac{B_{n}(R,Z)}{B_{0}(R,Z)}.
\end{equation}
\end{enumerate}
%

\foilhead{Status}
\begin{enumerate}
\item Computation of radial fluxes
\begin{enumerate}
\item Orbit frequencies
\begin{enumerate}
\item Can compute orbit frequencies $\omega_{b}$ and $\Omega^{\varphi}$
from given starting $R,Z$
\item TODO: Interpolation over action space for all classes
\end{enumerate}
\item Level set integrals in phase-space
\begin{enumerate}
\item 1D case works with level set methods to trace ``forbidden region''
\item TODO: 2D case works in prototype, but not yet applied
\end{enumerate}
\item Box counting method
\begin{enumerate}
\item Case for splitting orbit between boxes works
\item TODO: Integration of $\partial_{mn}r$ term with finite differences
\end{enumerate}
\end{enumerate}
\item Hamiltonian perturbation
\begin{enumerate}
\item Computation of $H_{n}(R,Z)$ looks correct
\item Bounce averages are working
\item TODO: Test exponential term for canonical harmonics
\end{enumerate}
\end{enumerate}
%

\foilhead{Missing ingredients}
\begin{enumerate}
\item Routine to compute starting position from invariants $(J_{\perp},p_{\varphi},H)$
@Sergei
\item Translation Boozer-cylindrical coordinates @Artem
\item Bounding region for invariants @Sergei, sub\_potato.f90 (last closed
surface)
\item Interpolation of orbit frequencies over invariants @Sergei
\item Comparison with NEO-RT v1 and/or NEO-2, e.g. AUG 30835, $n=2$ and
$B_{n}=5\cdot10^{-3}B_{0}\cos\vartheta$ or real Boozer @Chris,Rico
\item Literature study: see also Lausanne, Monte-Carlo code VENUS
\end{enumerate}
~

Some day: optimize, e.g. save some bounce evaluations


\end{document}
